\documentclass[12pt]{article}
\usepackage{usecases}
\begin{document}
\begin{usecase}

\addtitle{Use Case 1}{Create Project} 

%Scope: the system under design
\addfield{Scope:}{System-wide}	
%Primary Actor: Calls on the system to deliver its services.
\addfield{Primary Actor:}{Administrator}

%Stakeholders and Interests: Who cares about this use case and what do they want?
\additemizedfield{Stakeholders and Interests:}{
	\item Administrator: Needs to be able to create and modify projects
	\item Project Leader: Needs to modify projects and add activities to projects
	\item Developer: Needs to be able to see projects
	
}
\addfield{Description:}{Makes it possible to create a project}
%Main Success Scenario: A typical, unconditional happy path scenario of success.
\addscenario{Main Scenario:}{
	\item Gives the administrator the possibility to create projects
	\item The administrator will be able to create with a given name and project leader.}

%Extensions: Alternate scenarios of success or failure.
\addscenario{Alternative Scenarios:}{
	\item[1.a] Invalid login data:
		\begin{enumerate}
		\item[1.] System shows failure message
		\item[2.] System returns to step 1
		\end{enumerate}
	\item[2.a] No project leader assigned:
		\begin{enumerate}
		\item[1.] If project leader is not assigned, it will create the project with no project leader.
		\item[2.] Administrator will be able to set the project leader by modifying the project
		\end{enumerate}
	\item[2.b] No project name assigned:
		\begin{enumerate}
		\item[1.] If project name is not assigned, a default name will be assigned
		\item[2.] Administrator or project leader will be able to change the name, by modifying the project.
		\end{enumerate}
		}
\end{usecase}
\end{document}